\documentclass{article}
\usepackage[UTF8]{ctex}
\usepackage{listings}
\usepackage{url}
\title{美国联邦储备银行 爬虫 开发手册}
\author{Snail XHR}
\date{2021.7.14}
\begin{document}
    \maketitle
    \newpage
    \renewcommand{\contentsname}{目录}
    \tableofcontents
    \newpage
    \noindent\section{关于本爬虫}
    本爬虫用于获取美国联邦储备银行网站上的内容。
    (\url{https://www.federalreserve.gov/newsevents.htm})。
    \par
    爬虫获取的是新闻以及演讲稿。
    \par
    备注:该网站有三个栏目,分别是新闻、演讲、证词。该爬虫没有爬取证词。 
    
    \noindent\section{实现原理}
    爬虫首先爬取新闻与活动页面中新闻稿、演讲稿的地址。对于每篇文章,分别获取其标题及内文,并写入文件。
    \par
    据观察,新闻、演讲的链接都在一对 class 属性为 news news\_\_title 的 p 标签中。
    我们初步的解决方案是,先以空格为分割符,将该 p 标签的源码进行切片,在提取其中的链接。此时,需要保证该标签中没有除所需内容以外的链接。
    
    \par
    接下来,对于每篇新闻,其标题被包含在一对 class 属性为 title 的 h3 标签中。
    所有的正文则都被包含在了一个 class 属性为 col-xs-12  col-sm-8  col-md-8 的 div 标签中。
    我们只需提取这两个标签中的内容即可。
    \noindent\section{实现方法}
    我们选择了用 Python 实现爬虫,依赖的库包括 Beautiful Soup4 , requests , re(正则表达式) , lxml(XML解析)。
    我们一般使用 requests 库下载页面,并使用 Beautiful Soup4 过滤结果。
    最后,用正则表达式剔除 HTML 标签。
    \noindent\section{函数说明}
    \subsection{get\_news\_list(base\_url)}
    该函数用于获取所有新闻与演讲稿的 URL。
    \par
    参数:你应当提供 新闻与活动 页面的 URL ,目前为
    \url{https://www.federalreserve.gov/newsevents.htm}。
    \par
    该函数将会返回一个列表,包含所有新闻、演讲稿的 URL。
    \subsection{get\_news(news\_url)}
    该函数用于获取一篇文章的标题和正文。
    \par
    参数:你应当提供一个 URL ,即是你需要获取的新闻的网址。
    \par
    该函数将会返回一个字典,其中包含两个键:title 和 news 。
    title 键对应的值是该新闻的标题的 HTML,news 键则对应正文的 HTML。
    值得注意的是,此时返回的数据是 HTML 代码,而非纯文本。
    \subsection{delete\_html\_tag(string)}
    该函数用于删除一个字符串中的 HTML 标签,注意,它只是寻找尖括号,并将其中的内容删除而以。
    因 HTML 标签形如:<tag> text </tag>,所以我们可以很轻易地删除 HTML 标签。
    该函数的原理是正则替换。使用的正则表达式:
    \begin{lstlisting}
        <(\ S*?)*[^>]*>.*?|<.*? />
    \end{lstlisting}
    \subsection{write\_into\_file(file\_name,title,article)}
    该函数用于将一篇新闻写入文件。
    \par
    参数:包含三个参数,file\_name 用于是目标文件名。
    title 则是新闻的标题,将在文件的最上方打印,并在其下留出空行。
    article 是正文,将被原样输出至文件中。
    \par
    请注意,该函数纯粹只将参数写入文件,不会对其内容作改变。
    请保证你的参数与你的期望格式相同。
\end{document}