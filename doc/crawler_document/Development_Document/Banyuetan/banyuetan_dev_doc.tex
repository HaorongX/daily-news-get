\documentclass{article}
\usepackage[UTF8]{ctex}
\usepackage{listings}
\usepackage{url}
\title{半月谈·今日谈 爬虫 开发手册}
\author{Snail XHR}
\date{2021.7.14}
\begin{document}
    \maketitle
    \newpage
    \renewcommand{\contentsname}{目录}
    \tableofcontents
    \newpage
    \noindent\section{关于本爬虫}
    本爬虫用于获取半月谈网上今日谈的内容
    (\url{http://www.banyuetan.org/byt/jinritan/index.html})。
    \par
    爬虫获取的是今日谈中的 Top10 要闻。
    
    \noindent\section{实现原理}
    爬虫首先爬取今日谈页面中 Top10 要闻的地址。对于每篇文章,分别获取其标题及内文,并写入文件。
    \par
    据观察,Top10 要闻 一栏在一对 class 属性为 title2 的 div 标签中。进一步的,我们发现
    ,所有在这一栏中的 a 标签(指向文章的链接)的 class 属性都为 special1。
    因此,在提取新闻地质这一阶段,我们只需要在整个网页中寻找 class 为 special1 的 a 标签。
    \par
    接下来,对于每篇新闻,我们发现整个页面中有且只有一个 h1 标签,该标签中即是新闻标题。
    所有的正文则都被包含在了一个 class 属性为 detail\_content 的 div 标签中。
    我们只需提取这两个标签中的内容即可。
    \noindent\section{实现方法}
    我们选择了用 Python 实现爬虫,依赖的库包括 Beautiful Soup4 , requests , re(正则表达式) , lxml(XML解析)。
    我们一般使用 requests 库下载页面,并使用 Beautiful Soup4 过滤结果。
    最后,用正则表达式剔除 HTML 标签。
    \noindent\section{函数说明}
    \subsection{get\_news\_list(base\_url)}
    该函数用于获取 "Top10要闻" 的 URL。
    \par
    参数:你应当提供 "Top10要闻" 所处页面的 URL ,目前为
    \url{http://www.banyuetan.org/byt/jinritan/index.html}。
    \par
    该函数将会返回一个列表,包含 "Top10要闻" 的 URL。
    \subsection{get\_news(news\_url)}
    该函数用于获取一篇新闻的标题和正文。
    \par
    参数:你应当提供一个 URL ,即是你需要获取的新闻的网址。
    \par
    该函数将会返回一个字典,其中包含两个键:title 和 news 。
    title 键对应的值是该新闻的标题的 HTML,news 键则对应正文的 HTML。
    值得注意的是,此时返回的数据是 HTML 代码,而非纯文本。
    \subsection{delete\_html\_tag(string)}
    该函数用于删除一个字符串中的 HTML 标签,注意,它只是寻找尖括号,并将其中的内容删除而以。
    因 HTML 标签形如:<tag> text </tag>,所以我们可以很轻易地删除 HTML 标签。
    该函数的原理是正则替换。使用的正则表达式:
    \begin{lstlisting}
        <(\ S*?)*[^>]*>.*?|<.*? />
    \end{lstlisting}
    \subsection{write\_into\_file(file\_name,title,article)}
    该函数用于将一篇新闻写入文件。
    \par
    参数:包含三个参数,file\_name 用于是目标文件名。
    title 则是新闻的标题,将在文件的最上方打印,并在其下留出空行。
    article 是正文,将被原样输出至文件中。
    \par
    请注意,该函数纯粹只将参数写入文件,不会对其内容作改变。
    请保证你的参数与你的期望格式相同。
\end{document}